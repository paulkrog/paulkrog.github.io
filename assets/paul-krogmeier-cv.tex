%% start of file `template.tex'.
%% Copyright 2006-2013 Xavier Danaux (xdanaux@gmail.com).
%
% This work may be distributed and/or modified under the
% conditions of the LaTeX Project Public License version 1.3c,
% available at http://www.latex-project.org/lppl/.

% TODO: add senior design, 337, and 437 projects

\documentclass[10pt,a4paper,sans]{moderncv}        % possible options include font size ('10pt', '11pt' and '12pt'), paper size ('a4paper', 'letterpaper', 'a5paper', 'legalpaper', 'executivepaper' and 'landscape') and font family ('sans' and 'roman')

% modern themes
\moderncvstyle{banking}                            % style options are 'casual' (default), 'classic', 'oldstyle' and 'banking'
\moderncvcolor{black}                                % color options 'blue' (default), 'orange', 'green', 'red', 'purple', 'grey' and 'black'
%\renewcommand{\familydefault}{\sfdefault}         % to set the default font; use '\sfdefault' for the default sans serif font, '\rmdefault' for the default roman one, or any tex font name
%\nopagenumbers{}                                  % uncomment to suppress automatic page numbering for CVs longer than one page

% character encoding
\usepackage[utf8]{inputenc}                       % if you are not using xelatex ou lualatex, replace by the encoding you are using
%\usepackage{CJKutf8}                              % if you need to use CJK to typeset your resume in Chinese, Japanese or Korean

% adjust the page margins
\usepackage[scale=0.75]{geometry}
%\setlength{\hintscolumnwidth}{3cm}                % if you want to change the width of the column with the dates
%\setlength{\makecvtitlenamewidth}{10cm}           % for the 'classic' style, if you want to force the width allocated to your name and avoid line breaks. be careful though, the length is normally calculated to avoid any overlap with your personal info; use this at your own typographical risks...

\usepackage{import}

% personal data
\name{Paul}{Krogmeier}
\title{CV}                               % optional, remove / comment the line if not wanted
\address{122 Circle Lane Drive, West Lafayette, IN, 47906}{}{}% optional, remove / comment the line if not wanted; the "postcode city" and and "country" arguments can be omitted or provided empty
\phone[mobile]{+1 765 404 6297}                   % optional, remove / comment the line if not wanted
%\phone[fixed]{01234 123456}                    % optional, remove / comment the line if not wanted
%\phone[fax]{+3~(456)~789~012}                      % optional, remove / comment the line if not wanted
\email{pkrogmei@purdue.edu}                               % optional, remove / comment the line if not wanted
\homepage{paulkrog.github.io}                         % optional, remove / comment the line if not wanted
%\extrainfo{additional information}                 % optional, remove / comment the line if not wanted
%\photo[64pt][0.4pt]{picture}                       % optional, remove / comment the line if not wanted; '64pt' is the height the picture must be resized to, 0.4pt is the thickness of the frame around it (put it to 0pt for no frame) and 'picture' is the name of the picture file
%\quote{Some quote}                                 % optional, remove / comment the line if not wanted

% to show numerical labels in the bibliography (default is to show no labels); only useful if you make citations in your resume
%\makeatletter
%\renewcommand*{\bibliographyitemlabel}{\@biblabel{\arabic{enumiv}}}
%\makeatother
%\renewcommand*{\bibliographyitemlabel}{[\arabic{enumiv}]}% CONSIDER REPLACING THE ABOVE BY THIS

% bibliography with mutiple entries
%\usepackage{multibib}
%\newcites{book,misc}{{Books},{Others}}
%----------------------------------------------------------------------------------
%            content
%----------------------------------------------------------------------------------
\begin{document}
%\begin{CJK*}{UTF8}{gbsn}                          % to typeset your resume in Chinese using CJK
%-----       resume       ---------------------------------------------------------
\makecvtitle

\small{ Pursuing master's degree in computer engineering from Purdue
  University. Seeking PhD opportunities in computer science. }

\section{Experience}

\vspace{4pt}

\begin{itemize}


\item{\cventry{Jun 2017}{Oregon Progamming Languages Summer
      School}{OPLSS 2017}{Eugene, Oregon}{}{\vspace{3pt}
      \parbox{0.75\linewidth}{Attended research lectures from experts
        in Programming Languages and Formal Methods. Participated in
        hands-on sessions for learning about current research software
        and techniques: Idris, PLT Redex, Concurrent C0}}}

\item{\cventry{Sep 2016--Dec 2016}{Purdue E-lab}{Deep Learning}{West
      Lafayette}{}{\vspace{3pt} \parbox{0.75\linewidth}{Used Torch7
        deep learning framework to find solutions to reinforcement
        learning problems.}}}

\vspace{4pt}

\item{\cventry{May 2016--Jul 2016}{APOLO computing group}{Software for
      HPC cluster administration}{Medellin, Colombia}{}{\vspace{3pt}
      \parbox{0.75\linewidth}{Developed software to produce client
        usage reports for a Linux Rocks cluster administrative team.}
      \begin{itemize}
      \item \parbox{0.75\linewidth}{Wrote python scripts to query
          cluster load and usage patterns and to present gathered
          information clearly and succinctly}
    \item \parbox{0.75\linewidth}{Interfaced with TORQUE and SLURM
        resource management systems}
      \end{itemize}
}}

\vspace{4pt}

\item{\cventry{May 2014--Jul 2014}{Purdue OADA undergraduate research
      team}{Embedded systems programming}{West
      Lafayette}{}{\vspace{2pt} \parbox{0.75\linewidth}{Developed
        software for a wireless, embedded semi-truck weight sensing
        application. The goal was to provide a way for truck drivers
        to quickly learn the weight of their load through an app
        interface that communicated wirelessly with an embedded
        circuit board.}
      \begin{itemize}
      \item \parbox{0.75\linewidth}{Interfaced Nordic nRF51822 SoC to
          air pressure sensors over I2C}
      \item \parbox{0.75\linewidth}{Programmed communication between
          Andriod application and SoC using Bluetooth Low Energy
          stack}
      \end{itemize}
}}

\vspace{4pt}

\end{itemize}


\section{Education}

\vspace{3pt}

\subsection{Graduate}

\vspace{3pt}

\begin{itemize}

\item{\cventry{2016--present}{ M.S. Computer Engineering}{Purdue
      University}{West Lafayette}{\textbf{GPA: 3.98}}{}}

  \begin{itemize}
  \item{\textbf{Teaching Assistant:}}
    \begin{itemize}
    \item ECE 369 -- Discrete Math
    \end{itemize}
  \item{
      \textbf{Masters Project ('Formalization of Fiat in Coq')}
      \begin{itemize}
      \item \parbox{0.75\linewidth}{Developing proof of type safety
          for the Fiat specification language in Coq. Additionally,
          exploring the potential for synthesis of performant Haskell
          code from high level specifications. }
      \end{itemize}
    }
  \end{itemize}

% \item{\textbf{Masters Project (Ongoing):} \textit{'Semantically-driven
%       SAT solving'}

% \vspace{2pt}


% \newpage

\end{itemize}

\subsection{Undergraduate}

\vspace{3pt}

\begin{itemize}

\item{\cventry{2012--2016}{ B.S. Computer Engineering}{Purdue
      University}{West Lafayette}{\textbf{GPA: 4.0}}{}}

\item{\cventry{Spring 2015}{Study Abroad}{EAFIT University}{Medellin,
      Colombia}{Compilers and Operating Systems courses}{}}  % arguments 3 to 6 can be left empty

\end{itemize}

\vspace{2pt}

% \subsection{Course Projects}

% \vspace{3pt}

% \begin{itemize}

% \item{\cventry{Spring 2016}{ECE 477}{Digital Systems Senior Design}{}{}{\vspace{2pt} Designed, implemented, and machined a
%       digital, MIDI-outputting trombone.
%       \begin{itemize}
%       \item UART communication with MIDI keyboard
%       \item Captured player input with various sensors
%       \item Designed, soldered, programmed, and debugged printed
%         circuit board.
%       \end{itemize}
%     }}

% \item{\cventry{Spring 2016}{ECE 437}{Computer
%       Architecture}{}{}{\vspace{2pt} Programmed dual-core 5-stage
%       pipelined processor in verilog.
%       \begin{itemize}
%       \item Implemented a subset of the MIPS ISA.
%       \item Implemented cache coherency protocol for instruction and
%         data caches.
%       \item Implemented branch lookahead buffer.
%       \end{itemize}
%     }}

% \end{itemize}

\subsection{Courses}

\begin{minipage}[t]{0.45\linewidth}
  \raggedright
  \textbf{Graduate}
  \begin{tabular}{ll}
    CE 642 -- Information Theory and Source Coding \\
    CS 590 -- Reasoning About Programs (Audit) \\
    CE 573 -- Compilers and Translator Systems \\
    CE 608 -- Computational Models and Methods \\
    CE 600 -- Probabilities and Random Processes \\
    CS 565 -- Programming Languages \\
    CS 590 -- Artificial Intelligence and Causal Inference \\
    CS 584 -- Theory of Computation and Complexity \\
    CS 573 -- Data Mining \\
  \end{tabular}
\end{minipage}
\hfill
\begin{minipage}[t]{0.45\linewidth}
  \raggedright
  \textbf{Undergraduate}
  \begin{tabular}{ll}
    CE 368 -- Algorithms and Data Structures \\
    CE 369 -- Discrete Math \\
    CE 364 -- Python and Bash Scripting Lab \\
    CE 337 -- ASIC Design Laboratory \\
    CE 437 -- Computer Architecture \\
    CE 477 -- Digital Systems Senior Design \\
  \end{tabular}
\end{minipage}

\section{Technical and Personal skills}

\vspace{4pt}

\begin{itemize}

\item \textbf{Programming Languages:} \\
  Proficient in: C/C++, Python, Matlab, and Verilog \\ Basic experience with: SML, Idris, Racket,
  x86 ISA, MIPS ISA, Java, Lisp, Jekyll/HTML/(S)CSS

\vspace{4pt}

\item \textbf{Research Software:} Coq, Rosette, Sketch, Fiat

\vspace{4pt}

\item \textbf{Natural Languages:} Fluent in Spanish, German, and English (native)

\vspace{4pt}

\item \textbf{Other:} Lead alto saxophone player in Purdue University
  Jazz Band

\end{itemize}

\section{Awards}

\vspace{4pt}

\begin{itemize}


\item \textbf{Purdue Ross Fellowship:} May 2016
\vspace{4pt}
\item \textbf{100K Strong in the Americas Scholarship:} August 2014

\end{itemize}


% Publications from a BibTeX file without multibib
%  for numerical labels: \renewcommand{\bibliographyitemlabel}{\@biblabel{\arabic{enumiv}}}% CONSIDER MERGING WITH PREAMBLE PART
%  to redefine the heading string ("Publications"): \renewcommand{\refname}{Articles}
\nocite{*}
\bibliographystyle{plain}
\bibliography{publications}                        % 'publications' is the name of a BibTeX file

% Publications from a BibTeX file using the multibib package
%\section{Publications}
%\nocitebook{book1,book2}
%\bibliographystylebook{plain}
%\bibliographybook{publications}                   % 'publications' is the name of a BibTeX file
%\nocitemisc{misc1,misc2,misc3}
%\bibliographystylemisc{plain}
%\bibliographymisc{publications}                   % 'publications' is the name of a BibTeX file

%-----       letter       ---------------------------------------------------------

\end{document}


%% end of file `template.tex'.