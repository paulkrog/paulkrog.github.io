\documentclass[sigchi,12pt,a4paper,sans,nonacm]{acmart}

\usepackage[utf8]{inputenc}
\usepackage{xcolor}
\usepackage{lipsum}
\usepackage{setspace}
\usepackage{enumitem}
\usepackage{titlesec}
% \usepackage{libertinus}  % Loads the Linux Libertine font
\titleformat{\section}[block]{\normalsize\bfseries\sffamily}{\thesection}{1em}{}

\definecolor{zgreen}{HTML}{2F4F3F}
\newcommand{\myh}[3][zgreen]{\href{#2}{\color{#1}{#3}}}


\pagenumbering{gobble}
\def\shorttitle{}
\setlist[enumerate]{leftmargin=10pt}

\begin{document}

% \setlength{\abovedisplayskip}{0pt}
% \setlength{\belowdisplayskip}{0pt}
% \setlength{\abovedisplayshortskip}{0pt}
% \setlength{\belowdisplayshortskip}{0pt}

{\centering {\Large {\bf Paul Krogmeier }}\par}

\vspace{0.2in}
% \renewcommand{\arraystretch}{0.8}

\noindent \myh{mailto: paulmk2@illinois.edu}{paulmk2@illinois.edu}

\noindent \myh{https://paulkrog.github.io}{https://paulkrog.github.io}

\section*{\MakeUppercase{Education}}

\vspace{0.2in}

% \leftskip 0.3in
\renewcommand{\arraystretch}{0.9}

\begin{tabular*}{\textwidth}{l@{\extracolsep{\fill}}r}
  \textbf{University of Illinois Urbana-Champaign} & Expected \\
  \textit{Ph.D. in Computer Science (advisor: Madhusudan
  Parthasarathy).} &  Fall 2024 \\
  Ph.D. Thesis: Theory and Algorithms for Symbolic Learning. &
\end{tabular*}

\vspace{0.2in}
\noindent
\begin{tabular*}{\textwidth}{l@{\extracolsep{\fill}}r}
  \textbf{Purdue University} &  \\
  \textit{M.S. in Computer Engineering (advisor: Benjamin Delaware).} & 2016{--}2018 \\
  M.S. Thesis: A Core Calculus for Data Refinement. & \\
  \textit{B.S. in Computer Engineering (with highest distinction).} & 2012{--}2016
\end{tabular*}

\vspace{0.1in}

\section*{\MakeUppercase{Research Interests}}
\label{sec:research-interests}

\vspace{0.2in}
\begin{enumerate}
\item[] \begin{singlespace}
\noindent
My interests are in the foundations of \textbf{symbolic learning and
  reasoning}, with a focus on the problem of learning symbolic
concepts that describe \textbf{structured data} like sequences, trees,
graphs, or states of computer programs. This encompasses program
synthesis from examples as well as learning classifiers expressed in
logic. Recently, I have been exploring how to \textbf{synthesize
  domain-specific languages} to support efficient few-shot symbolic
learning.
\end{singlespace}
\end{enumerate}


\section*{\MakeUppercase{Awards}}
\vspace{0.2in}

\noindent
\begin{tabular*}{\textwidth}{l@{\extracolsep{\fill}}r}
  ACM SIGPLAN Distinguished Paper Award at OOPSLA & 2023 \\
  ACM SIGPLAN Distinguished Paper Award at POPL & 2022 \\
  Illinois Wing Kai Cheng Fellowship & 2018 \\
  Purdue Ross Fellowship & 2016 \\
\end{tabular*}

\vspace{0.1in}
\section*{\MakeUppercase{Refereed Conference Publications}}
\vspace{0.2in}

\begin{enumerate}[itemsep=16pt]
% \item[] \begin{tabular*}{1.0\linewidth}[l]{l}
%     Paul Krogmeier and P. Madhusudan. \\
% \myh{https://paulkrog.github.io/papers/MetaTheorem.pdf}{\underline{\smash{Languages
%           with Decidable Learning: a Meta-theorem.}}} \\
%     \textbf{Distinguished paper}, OOPSLA 2023.
%   \end{tabular*}
\item[] Paul Krogmeier and P. Madhusudan. 2023. Languages with
  Decidable Learning: A Meta-theorem. Proc. ACM Program. Lang. 7,
  OOPSLA1, Article 80 (April 2023), 29
  pages. \myh{https://doi.org/10.1145/3586032}{https://doi.org/10.1145/3586032}
  \\ \textbf{ACM SIGPLAN Distinguished Paper Award.}
% \item[] \begin{tabular*}{1.0\linewidth}[l]{l}
%     Paul Krogmeier$^*$, Zhengyao Lin$^*$, Adithya Murali$^*$, and P. Madhusudan. \\
%     \myh{https://dl.acm.org/doi/pdf/10.1145/3563348}{\underline{\smash{Synthesizing Axiomatizations using Logic Learning.}}} \\
%     OOPSLA 2022.
%   \end{tabular*}
\item[] Paul Krogmeier$^*$, Zhengyao Lin$^*$, Adithya Murali$^*$, and
  P. Madhusudan. 2022. Synthesizing axiomatizations using logic
  learning. Proc. ACM Program. Lang. 6, OOPSLA2, Article 185 (October
  2022), 29
  pages. \myh{https://doi.org/10.1145/3563348}{https://doi.org/10.1145/3563348}
% \item[] \begin{tabular*}{1.0\linewidth}[l]{l}
%         Adithya Murali, Atharva Sehgal, Paul Krogmeier, and P. Madhusudan. \\
%         \myh{https://doi.org/10.24963/ijcai.2022/466}{\underline{\smash{Composing Neural Learning and Symbolic Reasoning with an Application to Visual Discrimination.}}} \\
%         IJCAI 2022.
%   \end{tabular*}
\item[]
\item[] Adithya Murali, Atharva Sehgal, Paul Krogmeier,
  P. Madhusudan. Composing Neural Learning and Symbolic Reasoning with
  an Application to Visual Discrimination. Proceedings of the
  Thirty-First International Joint Conference on Artificial
  Intelligence Main Track (IJCAI). Pages
  3358-3365.\\ \myh{https://doi.org/10.24963/ijcai.2022/466}{https://doi.org/10.24963/ijcai.2022/466}
% \item[] \begin{tabular*}{1.0\linewidth}[l]{l}
%         Paul Krogmeier and P. Madhusudan. \\
%         \myh{https://doi.org/10.1145/3498671}{\underline{\smash{Learning Formulas in Finite Variable Logics.}}} \\
%         \textbf{Distinguished paper}, POPL 2022.
%   \end{tabular*}
\item[] Paul Krogmeier and P. Madhusudan. 2022. Learning formulas in
  finite variable logics. Proc. ACM Program. Lang. 6, POPL, Article 10
  (January 2022), 28
  pages. \myh{https://doi.org/10.1145/3498671}{https://doi.org/10.1145/3498671}
  \\ \textbf{ACM SIGPLAN Distinguished Paper Award.}
% \item[] \begin{tabular*}{1.0\linewidth}[l]{l}
%         Gilles Barthe, Rohit Chadha, Paul Krogmeier, Aravinda Sistla, Mahesh Viswanathan. \\
%         \myh{https://dl.acm.org/doi/abs/10.1145/3434289}{\underline{\smash{Deciding Accuracy of Differential Privacy
%     Schemes.}}} \\
%         POPL 2021.
%   \end{tabular*}
\item[] Gilles Barthe, Rohit Chadha, Paul Krogmeier, A. Prasad Sistla,
  and Mahesh Viswanathan. 2021. Deciding accuracy of differential
  privacy schemes. Proc. ACM Program. Lang. 5, POPL, Article 8
  (January 2021), 30 pages. \myh{https://doi.org/10.1145/3434289}{https://doi.org/10.1145/3434289}
% \item[] \begin{tabular*}{1.0\linewidth}[l]{l}
%         Paul Krogmeier, P. Madhusudan, Umang Mathur, Adithya
%   Murali, Mahesh Viswanathan. \\
%         \myh{https://paper.springer.com/chapter/10.1007/978-3-030-53291-8_32}{\underline{\smash{Decidable Synthesis of Programs with Uninterpreted
%     Functions.}}} \\
%         CAV 2020.
%   \end{tabular*}
\item[] Krogmeier, P., Mathur, U., Murali, A., Madhusudan, P.,
  Viswanathan, M. (2020). Decidable Synthesis of Programs with
  Uninterpreted Functions. In: Lahiri, S., Wang, C. (eds) Computer
  Aided Verification. CAV 2020. Lecture Notes in Computer Science,
  vol 12225. Springer,
  Cham. \\ \myh{https://doi.org/10.1007/978-3-030-53291-8_32}{https://doi.org/10.1007/978-3-030-53291-8\_32}
% \item[] \begin{tabular*}{1.0\linewidth}[l]{l}
%         Umang Mathur, Adithya Murali, Paul Krogmeier,
%   P. Madhusudan, and Mahesh Viswanathan. \\
%         \myh{https://doi.org/10.1145/3371103}{\underline{\smash{Deciding Memory Safety for Single-pass Heap-manipulating
%     Programs.}}} \\
%         POPL 2019.
%   \end{tabular*}
\item[] Umang Mathur, Adithya Murali, Paul Krogmeier, P. Madhusudan,
  and Mahesh Viswanathan. 2019. Deciding memory safety for single-pass
  heap-manipulating programs. Proc. ACM Program. Lang. 4, POPL,
  Article 35 (January 2020), 29 pages. \myh{https://doi.org/10.1145/3371103}{https://doi.org/10.1145/3371103}
\end{enumerate}

\section*{\MakeUppercase{Workshop Publications}}
\label{sec:worksh-publ}
\vspace{0.1in}

\begin{enumerate}[itemsep=6pt]
\item[] \begin{tabular*}{1.0\linewidth}[l]{l} Paul Krogmeier,
          Steven Kidd, Benjamin Delaware.\\
          \myh{https://popl18.sigplan.org/details/CoqPL-2018/4/Towards-Context-Aware-Data-Refinement}{\underline{\smash{Towards Context-Aware Data Refinement.}}}
          CoqPL 2018.
        \end{tabular*}
\end{enumerate}

% \section*{\MakeUppercase{Work in Progress}}
% \label{sec:wip}
% \vspace{0.1in}

% \begin{enumerate}[itemsep=2pt]
% \item[] \begin{tabular*}{1.0\linewidth}[l]{l}
%           Paul Krogmeier and P. Madhusudan. \\
%           \myh{}{\underline{\smash{Synthesizing DSLs for Few-Shot
%           Learning.}}} \emph{In preparation.} \\
%           Algorithms for synthesizing domain-specific languages that
%           can be learned from few examples. % \\ \emph{In preparation}.
%   \vspace{0.1in}
%         \end{tabular*}
% \item[] \begin{tabular*}{1.0\linewidth}[l]{l}
%           Paul Krogmeier. \\
%           \myh{}{\underline{\smash{Computing with Abstractions.}}}
%   \emph{In preparation.} \\
%           A new model of computation to study how
%           abstractions emerge in an evolving computation. \\
%           % \emph{In preparation}.
%         \end{tabular*}
% \end{enumerate}

\section*{\MakeUppercase{Teaching}}
\label{sec:teaching}
\vspace{0.2in}

\begin{enumerate}[itemsep=6pt]
\item[]
  \begin{tabular*}{1.0\linewidth}[l]{l@{\extracolsep{\fill}}r}
    CS 421: Programming Languages and Compilers & \textbf{University
                                                  of Illinois} \\
    & Fall 2019, Fall 2020, \\ & Spring 2021 -- Fall 2023, %  \\ & Spring
                                                %              2022, Fall
                                                %              2022, \\
                                                % & Spring 2023, Fall
                                                %   2023,
    \\
                                                & Fall 2024
  \end{tabular*}
\item[]
    \begin{tabular*}{1.0\linewidth}[l]{l@{\extracolsep{\fill}}r}
    ECE 369: Discrete Mathematics for Computer Engineering &
                                                             \textbf{Purdue
                                                             University} \\
                        & Fall 2017
  \end{tabular*}
\end{enumerate}

\section*{\MakeUppercase{Invited Talks}}
\vspace{0.1in}

\begin{enumerate}[itemsep=6pt]
\item[] \begin{tabular*}{1.0\linewidth}[l]{l@{\extracolsep{\fill}}r}
  Learning Symbolic Concepts and Domain-specific Languages & MIT EECS, Apr 2024\\
                                                           & Houston CS, Apr 2024 \\
                                                           & Purdue ECE/CS, Mar 2024
\end{tabular*}
\item[] \begin{tabular*}{1.0\linewidth}[l]{l@{\extracolsep{\fill}}r}
  Languages with Decidable Learning: a Meta-theorem & Boston U. CS, Mar 2023
\end{tabular*}
\item[] \begin{tabular*}{1.0\linewidth}[l]{l@{\extracolsep{\fill}}r}
  Learning Formulas in Finite-Variable Logics & St. Petersburg State University, Mar 2022
\end{tabular*}
\end{enumerate}

% \vspace{0.1in}
% \section*{\MakeUppercase{Invited Workshops}}
% \vspace{0.2in}

% \begin{enumerate}[itemsep=6pt]
% \item[] \begin{tabular*}{1.0\linewidth}[l]{l@{\extracolsep{\fill}}r}
%     \textbf{Dagstuhl seminar} & Fall 2019 \\
%     Logic and Learning &
%   \end{tabular*}
% \end{enumerate}
% \vspace{0.2in}

% \section*{\MakeUppercase{Teaching}}
% \label{sec:teaching}
% \vspace{0.2in}

% \begin{enumerate}[itemsep=6pt]
% \item[]
%   \begin{tabular*}{1.0\linewidth}[l]{l@{\extracolsep{\fill}}r}
%     CS 421: Programming Languages and Compilers & \textbf{University
%                                                   of Illinois} \\
%     & Fall 2019, Fall 2020, \\ & Spring 2021, Fall 2021 \\ & Spring
%                                                              2022, Fall
%                                                              2022, \\
%                                                 & Spring 2023, Fall
%                                                   2023, \\
%                                                 & Fall 2024
%   \end{tabular*}
% \item[]
%     \begin{tabular*}{1.0\linewidth}[l]{l@{\extracolsep{\fill}}r}
%     ECE 369: Discrete Mathematics for Computer Engineering &
%                                                              \textbf{Purdue
%                                                              University} \\
%                         & Fall 2017
%   \end{tabular*}
% \end{enumerate}

% \section*{\MakeUppercase{Service}}
% \label{sec:service}
% \vspace{0.1in}

% \begin{enumerate}[itemsep=2pt]
% \item[] \textbf{Journal Reviewer}
% \item[]
%   \begin{tabular*}{1.0\linewidth}[l]{l@{\extracolsep{\fill}}r}
%   Formal Methods in System Design (FMSD) & 2023
%   \end{tabular*}
% \item[] \textbf{Conference Reviewer}
% \item[]
%   \begin{tabular*}{1.0\linewidth}[l]{l@{\extracolsep{\fill}}r}
%   International Colloquium on Automata, Languages and
%   Programming (ICALP) & 2023
%   \end{tabular*}
% \item[]
%   \begin{tabular*}{1.0\linewidth}[l]{l@{\extracolsep{\fill}}r}
%     Logic in Computer Science (LICS) & 2022
%   \end{tabular*}
% \end{enumerate}

% \vspace{0.2in}
% \section*{\MakeUppercase{Mentoring}}
% \label{sec:mentoring}
% \vspace{0.1in}

% \begin{enumerate}[itemsep=6pt]
% \item[]
%   \begin{tabular*}{1.0\linewidth}[l]{l@{\extracolsep{\fill}}r}
%     SIGPLAN-M Graduate Student Mentor & 2023 -- Present
%   \end{tabular*}
% \end{enumerate}
% \vspace{0.2in}

% \section*{\MakeUppercase{Student Workshops}}
% \label{sec:student-workshops}
% \vspace{0.2in}

% \begin{enumerate}[itemsep=6pt]
% \item[]
%   \begin{tabular*}{1.0\linewidth}[l]{l@{\extracolsep{\fill}}r}
%     VMCAI Formal Methods Winter School & New Orleans, LA \\
%                                        & Jan 2020
%   \end{tabular*}
% \item[]
%   \begin{tabular*}{1.0\linewidth}[l]{l@{\extracolsep{\fill}}r}
%     SRI Formal Methods Summer School & Atherton, CA \\
%                                      & May 2019
%   \end{tabular*}
% \item[]
%   \begin{tabular*}{1.0\linewidth}[l]{l@{\extracolsep{\fill}}r}
%     Oregon Programming Languages Summer School & Eugene, OR \\
%                                                & Jun 2017
%   \end{tabular*}
% \end{enumerate}

% \section*{\MakeUppercase{Miscellany}}
% \label{sec:misc}
% \vspace{0.1in}

% \begin{enumerate}[itemsep=2pt]
% \item[] Native English speaker, fluent in Spanish, conversational in German.
% \item[] Jazz alto saxophonist with substantial performance and
%   teaching experience.
% \item[] Lover of snow, mountains, and skiing.
% \end{enumerate}

% \vspace{0.2in}

\end{document}
