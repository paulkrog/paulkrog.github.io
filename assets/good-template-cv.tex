\documentclass{article}
\usepackage{fullpage}
\usepackage{hyperref}

\usepackage{titlesec}
\titleformat{\section}{}{}{0em}{\large\bf}
\titleformat{\subsection}[runin]{}{}{0em}{\it}

\title{Max Willsey's CV}
\date{\today}
\author{Max Willsey}

\begin{document}

\begin{center}
  {\LARGE Max Willsey} \\
  \vspace{1em}
  \sf
  \href{mailto:mwillsey@cs.washington.edu}{mwillsey@cs.washington.edu} \\
  \href{http://mwillsey.com}{mwillsey.com}
\end{center}

\section{Education}

\subsection{University of Washington}
\hfill Sep 2016 -- present \\
Ph.D. Computer Science

\subsection{Carnegie Mellon University}
\hfill Aug 2012 -- May 2016 \\
B.S. Computer Science \\
Minor in Mathematics \\
University Honors and College Honors in Computer Science \\
Thesis: \href{http://maxwillsey.com/papers/cc0-thesis.pdf}{\textit{Design and Implementation of Concurrent C0}}

\section{Current Projects}

\subsection{Domain-Specific Reconfigurable Accelerators}
\hfill Sep 2016 -- present\\
with Vincent Lee, Luis Ceze, Rastislav Bodik, Alvin Cheung
\begin{itemize}
\item Exploring methods for designing and programming DSRAs using techniques like program synthesis
\item Automatically identifying building blocks that implement functionality across applications
\end{itemize}

\section{Past Projects}

\subsection{Concurrent C0 Design and Implementation}
\hfill Jan 2015 -- May 2016 \\
Senior Honors Thesis \\
Advisor: Frank Pfenning
\begin{itemize}
\item Worked on a concurrent extension to C0, a research project started as a well-defined subset of C
\item Used guarantees from session typing for efficient message passing implementation including intelligent scheduling decisions, lower memory impact, and deadlock free execution
\end{itemize}

\subsection{Abstractions for Concurrent Interactive Programs}
\hfill Aug 2014 -- Dec 2014 \\
Advisor: Umut Acar
\begin{itemize}
\item Worked on a functional programming for interaction, including an implementation in OCaml
\end{itemize}


\section{Teaching}

\subsection{Hardware/Software Interface} (CSE 351)
\hfill Dec 2016 -- Mar 2017\\
University of Washington

\subsection{Operating Systems} (15-410)
\hfill Aug 2015 -- May 2016 \\
Carnegie Mellon University
% Responsible for holding office hours and evaluating student projects, including code review of several kernel implementations

\section{Professional Experience}

\subsection{Apple}
\hfill May 2014/15 -- Aug 2014/15 \\
iOS Performance (2015): investigated and tested changes to scheduler \\
Siri Operations (2014): created a system for anomaly detection in logs

\subsection{SEI at Carnegie Mellon}
\hfill May 2013 -- Aug 2013 \\
Created a Twitter-like application for hundreds of users to coordinate training efforts in real time

\section{Publications}

Max Willsey, Rokhini Prabhu, and Frank Pfenning.
``Design and Implementation of Concurrent C0''.
Fourth International Workshop on Linearity, Electronic Proceedings in Theoretical Computer Science (EPTCS), June 2016

\section{Awards}

\subsection{Qualcomm Innovation Fellowship}
\hfill May 2017 \\
Program Synthesis for Domain Specific Reconfigurable Accelerators \\
with Vincent Lee, Luis Ceze, Rastislav Bodik, Alvin Cheung

\subsection{Exemplary Thesis}
\hfill May 2016 \\
Chosen by the senior thesis committee

\subsection{Andrew Carnegie Scholar}
\hfill Sep 2016 \\
40 seniors (of approx.~1500) selected by deans and dept.\ heads for leadership and academic excellence

\section{Coursework}


\begin{minipage}[t]{0.45\linewidth}
  \raggedright
  \textbf{University of Washington}
  \begin{tabular}{ll}
    548 & Computer Architecture \\
    507 & Computer-Aided Reasoning \\
    544 & Database Management Systems \\
  \end{tabular}
\end{minipage}
\hfill
\begin{minipage}[t]{0.45\linewidth}
  \raggedright
  \textbf{Carnegie Mellon University}
  \begin{tabular}{ll}
    15-417 & Higher Order Compilation  \\
    15-411 & Compiler Design                 \\
    15-312 & Programming Languages           \\
    15-410 & Operating Systems               \\
    15-451 & Algorithm Design/Analysis   \\
    15-213 & Computer Systems                \\
    21-484 & Graph Theory                    \\
    15-396 & Science of the Web              \\
  \end{tabular}
\end{minipage}
\end{document}