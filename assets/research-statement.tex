\documentclass{article}

\begin{document}

My primary interest lies at the intersection of formal logic and
learning. State of the art learning with neural networks models
intelligence behaviorally, i.e. a particular input-output
relationship. I hope to advance novel ideas that bridge the gap
between symbolic, interpretable reasoning, and the gradient-based
world of neural networks.  I am also interested in program synthesis
by example. It may be useful to study the problem of characterizing
programs by the smallest set of input-output examples that determines
them (when is a program the smallest program consistent with some
examples?). Finding such (hopefully) small sets would be useful for
the general program synthesis problem: for what classes of programs
and what kinds of examples is program synthesis from examples well
defined? Finally, I am interested in studying the synthesis problem
for programs that implement cryptographic protocols. Smart contracts,
which are small, may be amenable to synthesis techniques.

 % Such algorithms should be interpretable. By this I
% mean that a good learner should learn to represent knowledge in such a
% way that all hypotheses include an explanation of why a particular
% concept was learned. I think this is broadly useful, but rather vague
% criterion.
\end{document}